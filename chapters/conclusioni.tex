\chapter{Conclusioni e Sviluppi Futuri}
\label{cap:conclusioni}

\section{Consuntivo finale}
In questo studio, si sono analizzate diverse possibili strategie per alterare il traffico dati in connessioni \emph{QUIC},
con particolare attenzione sulle ritrasmissioni e sulle tecniche per aumentare artificialmente il consumo di dati.
\\\\
I risultati degli esperimenti condotti hanno confermato le ipotesi iniziali, rivelando una serie di metodi efficaci per aumentare il traffico dati. 
Si è evidenziato come un \emph{server} malevolo possa effettivamente incrementare il traffico dati, causando un aumento misurabile del consumo totale del \emph{client}.
In particolare, nel primo esperimento ignorando gli ACK, mentre nel secondo iniettando pacchetti aggiuntivi alla connessione normale. 
I risultati ottenuti suggeriscono che combinando le diverse strategie sperimentate si
potrebbe potenzialmente amplificare ulteriormente l'impatto sull'incremento del traffico. 
Ancora più significativi sono i risultati del terzo esperimento, dove si è evidenziato che anche senza il controllo diretto di \emph{client} o \emph{server}, 
un attaccante può manipolare il traffico dati, causando un aumento del 50\% nel volume di pacchetti trasmessi.

\section{Futuri Lavori}
~\\
\indent Nonostante il presente studio si sia concentrato principalmente sull'analisi delle vulnerabilità di \emph{QUIC} e sul loro potenziale sfruttamento per manipolare i sistemi di contabilizzazione del traffico mobile, questa ricerca si inserisce in un contesto più ampio che include anche un'analisi di \emph{MPTCP}, 
sebbene per ragioni di tempo l'attenzione è stata rivolta maggiormente su \emph{QUIC}. Un possibile sviluppo futuro è rappresentato da un'analisi più dettagliata della parte relativa a \emph{MPTCP}.
Sarebbe interessante replicare l'approccio utilizzato nell'esperimento uno di \emph{QUIC}, costruendo un server malevolo per \emph{MPTCP}, aspetto che per motivi di tempo non si è riusciti a implementare in questo studio.
\\\\
Un altro possibile sviluppo sarebbe l'estensione della sperimentazione in scenari reali, uscendo dagli ambienti controllati utilizzati finora. 
Condurre esperimenti in contesti reali permetterebbe di valutare l'effettivo impatto delle vulnerabilità individuate. 
Inoltre, questo approccio consentirebbe di analizzare le politiche di contabilizzazione delle ritrasmissioni adottate dai diversi operatori 
e di confrontarli per identificare eventuali differenze nel conteggio del consumo dati.
\\\\
Un approfondimento specifico su \emph{Multipath QUIC (MPQUIC)} sarebbe di grande interesse. 
Introdotto per la prima volta nel 2017 nel paper \emph{"Multipath QUIC: Design and Evaluation"} \cite{article:mpquic},
\emph{MPQUIC} è un'estensione del protocollo \emph{QUIC} che permette agli host di scambiare dati su reti multiple attraverso una singola connessione.
Data la sua natura di estensione di \emph{QUIC},
\emph{MPQUIC} potrebbe presentare vulnerabilità uniche o comportamenti di rete differenti, che meriterebbero un approfondimento.
