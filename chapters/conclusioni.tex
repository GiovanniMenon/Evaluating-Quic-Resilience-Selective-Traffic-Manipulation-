\chapter{Conclusioni e Sviluppi Futuri}
\label{cap:conclusioni}

\section{Consuntivo finale}

\section{Futuri Lavori}
~\\
\indent Nonostante il presente studio si sia concentrato principalmente sull'analisi delle vulnerabilità di QUIC e sul loro potenziale sfruttamento per manipolare i sistemi di contabilizzazione del traffico mobile, questa ricerca si inserisce in un contesto più ampio che include anche un'analisi di MPTCP, sebbene per ragioni di tempo ci si sia concentrati maggiormente su QUIC.
\\
Un possibile sviluppo futuro è rappresentato da un'analisi più dettagliata della parte relativa a MPTCP. Sarebbe particolarmente interessante applicare la costruzione di un server malevolo equivalente a quello dell'esperimento 1 di QUIC, aspetto che per motivi di tempo non si è riusciti a implementare in questo studio.
\\\\

Riprodurre gli esperimenti su MPTCP dedicandoci più tempo Questo studio si inserisce in un contesto più ampio che include anche l’analisi di
MPTCP, sebbene per ragioni di tempo ci si sia concentrati principalmente su QUIC.

Testare i diversi esperimenti in scenari reale e non sono in un ambiente di Testare,Sono numerose le cause che possono provocare la perdita di un pacchetto. Nel caso delle reti mobile ciò può accadere in punti diversi dell'infrasstruttura. Nel caso il segmento venga perso tra il client e la stazione base in \emph{RAN}\footnote{\gls{RAN}} è compito del \emph{link layer} occuparsi della ritrasmissione. Questo significa che, se il pacchetto venisse perso un numero di volte ma poi venisse ricevuto, il \emph{TCP} non rileverebbe alcun pacchetto perso ma solo un aumento del \emph{RTT}.
Questo studio si inserisce in un contesto più ampio che include anche l'analisi di \emph{MPTCP}, sebbene per ragioni di tempo ci si sia concentrati principalmente su \emph{QUIC}. 

Testare i diversi operatori. 

Approfondire quanto riguarda MPQUIC.