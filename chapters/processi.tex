\chapter{Processi e Metodi}
\label{cap:processi-metodologie}

\textit{\indent Questo capitolo fronisce in dettaglio l'ambiente di ricerca adottato, le tecnologie impiegate e degli esperimenti condotti. 
Fornisce inoltre tutte le informazioni necessarie per replicare gli esperimenti.}

\section{Ambiente}
~\\
\indent Questa sezione offre una panoramica completa dell'ambiente di lavoro e delle tecnologie impiegate nello sviluppo degli esperimenti descritti successivamente. 
Segue una descrizione degli strumenti utilizzati durante lo svolgimento del progetto (riassunte con le relative versioni nella Figura \ref{table-tecnologie}).
\\\\
Come premessa, la totalità del progetto è stata svolta su \textbf{\emph{Ubuntu}}\footnote{\url{https://ubuntu.com/}}. 
Questa scelta di sistema operativo è dovuta al vasto supporto di strumenti per l'analisi e lo sviluppo.
\\\\
Per l'analisi del traffico di rete, è stato ampiamente utilizzato \textbf{\emph{Wireshark}}\footnote{\url{https://www.wireshark.org/}}, 
uno strumento molto diffuso nel panorama del traffico di rete e che ha permesso di esaminare nel dettaglio il comportamento dei protocolli oggetti dello studio.
\\\\
La sperimentazione ha coinvolto l'uso di diversi \emph{browser web}, in particolare \textbf{\emph{Google Chrome}}\footnote{\url{https://www.google.com/intl/it_it/chrome/}} e \textbf{\emph{Firefox}}\footnote{\url{https://www.mozilla.org/it/firefox/}}. 
Questi due applicativi sono stati impiegati per testare e confrontare l'implementazione dei protocolli in diverse situazioni.
\\\\
L'ambiente di test è stato realizzato tramite \textbf{Oracle VM VirtualBox}\footnote{\url{https://www.virtualbox.org/}}, un diffuso ambiente di virtualizzazione. 
Permettendo di creare e gestire macchine virtuali specifiche per la creazione di possibili scenari.
\\\\
Per la condivisione e mantenimento del codice si è usato \textbf{GitHub}\footnote{\url{https://github.com/}} e \textbf{Git}.
In particolare si è rivelato vantaggioso in quanto ha permesso di lavorare efficientemente con \emph{fork}\footnote{\gls{fork}} di librerie pubbliche, consentendo una gestione dinamica delle modifiche e degli aggiornamenti del codice.

\subsection{Tecnologie Specifiche per QUIC}
~\\
\indent Per lo studio si è utilizzato \textbf{Quic-go}\footnote{\url{https://github.com/quic-go/quic-go}}, un'implementazione sviluppata in \emph{Go}\footnote{\gls{Go}} del protocollo \emph{QUIC}.
\emph{Quic-go} è un progetto \emph{open source} su \emph{GitHub} che aderisce rigorosamente alle specifiche del protocollo \emph{QUIC} 
definite negli \emph{RFC}\footnote{\gls{RFC}} 9000, 9001 e 9002. 
\\\\
\emph{Quic-go} non è l'unica versione disponibile del protocollo \emph{QUIC}.
Come illustrato nella Figura \ref{table-implementazioni-quic}, esistono numerose altre implementazioni, sia \emph{open source} che proprietarie, ciascuna cerca di rispettare rigorosamente le specifiche definite negli \emph{RFC}. 
\\\\
La scelta di utilizzare questa specifica implementazione di \emph{QUIC} è motivata dal suo utilizzo all'interno di \textbf{Caddy}\footnote{\url{https://caddyserver.com/}}, un \emph{web server} moderno e performante. \emph{Caddy} integra \emph{Quic-go} per offrire il supporto nativo del protocollo \emph{QUIC} e \emph{HTTP/3}.
Inoltre, si è utilizzato \textbf{xCaddy}, un \emph{tool} che consente di creare build personalizzate di \emph{Caddy}, adattandolo alle specifiche esigenze del progetto.

\begin{figure}[!h]
    \centering
    \begin{tabular}{|c|c|c|}
        \hline
        \textbf{Nome} & \textbf{Linguaggio} & \textbf{Licenza} \\
        \hline
        Chromium & C++ & BSD License \\
        \hline
        MsQuic & C & MIT License \\
        \hline
        QUIC Library (mvfst) & C++ & MIT License \\
        \hline
        LiteSpeed QUIC Library (lsquic) & C & MIT License \\
        \hline
        ngtcp2 & C & MIT License \\
        \hline
        Quiche & Rust & BSD-2-Clause License \\
        \hline
        quicly & C & MIT License \\
        \hline
        \textbf{quic-go} & \textbf{Go} & \textbf{MIT License} \\
        \hline
        Quinn & Rust & Apache License 2.0 \\
        \hline
        Neqo & Rust & Apache License 2.0 \\
        \hline
        aioquic & Python & BSD-3-Clause License \\
        \hline
        picoquic & C & BSD-3-Clause License \\
        \hline
        pquic & C & MIT License \\
        \hline
        QUANT & C & BSD-2-Clause License \\
        \hline
        quic & Haskell & BSD-3-Clause License \\
        \hline
        netty-incubator-codec-quic & Java & Apache License 2.0 \\
        \hline
        nodejs-quic & NodeJs & MIT License \\
        \hline
        s2n-quic & Rust & Apache License 2.0 \\
        \hline
        swift-quic & Swift & Apache License 2.0 \\
        \hline
        TQUIC & Rust & Apache License 2.0 \\
        \hline
        nginx & C & BSD-2-Clause License \\
        \hline
        HAProxy & C & GNU General Public License version 2 \\
        \hline
        kwik & Java & GNU Lesser General Public License version 3 \\
        \hline
    \end{tabular}
    \caption{\emph{Quic Source Code (da controllare)}}
    \label{table-implementazioni-quic}
\end{figure}

\subsection{Tecnologie Specifiche per MPTCP}
~\\
Per quanto riguarda \emph{MPTCP} si è fatto riferimento alla documentazione ufficiale \cite{site:mptcp-code} e si è usato \emph{Go} per sviluppare un \emph{web server} che implementa un \emph{socket MPTCP}.
\\\\
Per garantire che le applicazioni utilizzassero \emph{MPTCP} abbiamo usato \textbf{mptcpize}\footnote{\url{https://manpages.ubuntu.com/manpages/lunar/man8/mptcpize.8.html}}, un tool specifico che forza la creazione di \emph{socket MPTCP} al posto di quelli \emph{TCP}.


\begin{figure}[!h]
    \centering
    \begin{tabular}{|c|c|c|}
        \hline
        \textbf{Tipo} & \textbf{Nome} & \textbf{Versione} \\
        \hline
        Applicativo & \emph{Google Chrome} & 126.0.6478.126 \\
        \hline
        Applicativo & \emph{Firefox} & 127.0.2 \\
        \hline
        Applicativo & \emph{Github} &  \\
        \hline
        Applicativo & \emph{Git} & 2.43.0 \\
        \hline
        Applicativo & \emph{Oracle VM VirtualBox} & 7.0.20 \\
        \hline
        Applicativo & \emph{Wireshark} & 4.2.6 \\
        \hline
        Applicativo & \emph{Caddy} & 2.8.0 \\
        \hline
        Applicativo & \emph{xCaddy} & 0.4.2 \\
        \hline
        Applicativo & \emph{mptcpize} & 0.12 \\
        \hline
        Modulo & \emph{quic-go} &  0.43.1 \\
        \hline
        Sistema Operativo & \emph{Ubuntu} & 24.04 LTS \\
        \hline
        Linguaggio & \emph{Go} & 1.21 \\
        \hline
    \end{tabular}
    \caption{\emph{Tabella Tecnologie}}
    \label{table-tecnologie}
\end{figure}

\section{Esperimenti}

spiegazione degli esperimenti

\subsection{Esperimenti QUIC}

spiegazione degli esperimenti

\paragraph{Esperimento 1}
spiegazione degli esperimenti
\paragraph{Esperimento 2}
spiegazione degli esperimenti
\paragraph{Esperimento 3}
spiegazione degli esperimenti

\subsection{Esperimenti MPTCP}
\paragraph{Esperimento 1}

