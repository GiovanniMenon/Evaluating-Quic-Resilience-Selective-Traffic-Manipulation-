\chapter{Processi e Metodi}
\label{cap:processi-metodologie}

\textit{\indent Questo capitolo fronisce in dettaglio l'ambiente di ricerca adottato, le tecnologie impiegate e degli esperimenti condotti. 
Fornisce inoltre tutte le informazioni necessarie per replicare gli esperimenti.}

\section{Ambiente}
~\\
\indent Questa sezione offre una panoramica completa dell'ambiente di lavoro e delle tecnologie impiegate nello sviluppo degli esperimenti descritti successivamente. 
Segue una descrizione degli strumenti utilizzati durante lo svolgimento del progetto (riassunte con le relative versioni nella Figura \ref{table-tecnologie}).
\\\\
Come premessa, la totalità del progetto è stata svolta su \textbf{\emph{Ubuntu 24.04 LTS}}. 
Questa scelta di sistema operativo è dovuta al vasto supporto di strumenti per l'analisi e lo sviluppo.
\\\\
Per l'analisi del traffico di rete, è stato ampiamente utilizzato \textbf{\emph{Wireshark}}, 
uno strumento molto diffuso nel panorama del traffico di rete e che ha permesso di esaminare nel dettaglio il comportamento dei protocolli oggetti dello studio.
\\\\
La sperimentazione ha coinvolto l'uso di diversi \emph{browser web}, in particolare \textbf{\emph{Google Chrome}} e \textbf{\emph{Firefox}}. 
Questi due applicativi sono stati impiegati per testare e confrontare l'implementazione dei protocolli in diverse situazioni.
\\\\
L'ambiente di test è stato realizzato tramite \textbf{Oracle VM VirtualBox}, un diffuso ambiente di virtualizzazione. 
Permettendo di creare e gestire macchine virtuali specifiche per la creazione di possibili scenari.
\\\\
Per la condivisione e mantenimento del codice si è usato \textbf{GitHub} e \textbf{Git}.
In particolare si è rivelato vantaggioso in quanto ha permesso di lavorare efficientemente con \emph{fork}\footnote{\gls{fork}} di librerie pubbliche, consentendo una gestione dinamica delle modifiche e degli aggiornamenti del codice.

\subsection{Tecnologie Specifiche per QUIC}

CIASOID
\paragraph{Caddy}
\paragraph{Quic-go}
\subsection{Tecnologie Specifiche per MPTCP}
\paragraph{Mptcpize}

\begin{figure}[!h]
    \centering
    \begin{tabular}{|c|c|c|}
        \hline
        \textbf{Tipo} & \textbf{Nome} & \textbf{Versione} \\
        \hline
        Applicativo & \emph{Google Chrome} & Multipath Capable \\
        \hline
        Applicativo & \emph{Firefox} & Join Connection \\
        \hline
        Applicativo & \emph{Github} & Data Sequence Signal \\
        \hline
        Applicativo & \emph{Git} & Data Sequence Signal \\
        \hline
        Applicativo & \emph{Oracle VM VirtualBox} & Data Sequence Signal \\
        \hline
        Sistema Operativo & \emph{Ubuntu} & Change Subflow Priority \\
        \hline
        0x6 & \emph{MP\_FAIL} & Fallback \\
        \hline
        Linguaggio & \emph{Go} & Fast Close \\
        \hline
        Linguaggio & \emph{Python} & Subflow Reset \\
        \hline
    \end{tabular}
    \caption{\emph{Tabella Tecnologie}}
    \label{table-tecnologie}
\end{figure}

\section{Esperimenti}

spiegazione degli esperimenti

\subsection{Esperimenti QUIC}

spiegazione degli esperimenti

\paragraph{Esperimento 1}
spiegazione degli esperimenti
\paragraph{Esperimento 2}
spiegazione degli esperimenti
\paragraph{Esperimento 3}
spiegazione degli esperimenti

\subsection{Esperimenti MPTCP}
\paragraph{Esperimento 1}

