\chapter{Processi e Metodi}
\label{cap:processi-metodologie}

Questo capitolo fornisce in dettaglio l'ambiente di ricerca utilizzato, le tecnologie impiegate e descrive gli esperimenti condotti. 
Provvede a dare inoltre tutte le informazioni necessarie per replicare gli esperimenti.

\section{Ambiente}
~\\
\indent Questa sezione offre una panoramica completa dell'ambiente di lavoro e delle tecnologie impiegate nello sviluppo degli esperimenti descritti successivamente. 
Di seguito, viene riportata una descrizione degli strumenti utilizzati durante lo svolgimento del progetto (riassunte con le relative versioni nella Figura \ref{table-tecnologie}).
\\\\
La totalità del progetto è stata svolta su \textbf{\emph{Ubuntu}}\footnote{\url{https://ubuntu.com/}}. 
Questa scelta di sistema operativo è dovuta al vasto supporto di strumenti per l'analisi e lo sviluppo.
\\\\
Per l'analisi del traffico di rete, è stato utilizzato \textbf{\emph{Wireshark}}\footnote{\url{https://www.wireshark.org/}}, 
uno strumento molto diffuso nel panorama del traffico di rete e che ha permesso di esaminare nel dettaglio il comportamento dei protocolli oggetti dello studio.
\\\\
La sperimentazione ha coinvolto l'uso di diversi \emph{browser web}, in particolare \textbf{\emph{Google Chrome}}\footnote{\url{https://www.google.com/intl/it_it/chrome/}} e \textbf{\emph{Firefox}}\footnote{\url{https://www.mozilla.org/it/firefox/}}. 
Questi due applicativi sono stati impiegati per testare e confrontare l'implementazione dei protocolli in diverse situazioni.
\\\\
L'ambiente di test è stato realizzato tramite \textbf{Oracle VM VirtualBox}\footnote{\url{https://www.virtualbox.org/}}, un diffuso ambiente di virtualizzazione, che consente di creare e gestire più macchine virtuali.
\\\\
Per la condivisione e mantenimento del codice si è usato \textbf{GitHub}\footnote{\url{https://github.com/}} e \textbf{Git}.
In particolare si è rivelato vantaggioso in quanto ha permesso di lavorare efficientemente con \emph{fork}\footnote{\gls{fork}} di librerie pubbliche, consentendo una gestione dinamica delle modifiche e degli aggiornamenti del codice.

\subsection{Tecnologie Specifiche per QUIC}
~\\
\indent Per lo studio si è utilizzato \textbf{Quic-go}\footnote{\url{https://github.com/quic-go/quic-go}}, un'implementazione sviluppata in \emph{Go}\footnote{\gls{Go}} del protocollo \emph{QUIC}.
\emph{Quic-go} è un progetto \emph{open source} su \emph{GitHub} che aderisce rigorosamente alle specifiche del protocollo \emph{QUIC} 
definite negli \emph{RFC}\footnote{\gls{RFC}} 9000, 9001 e 9002 \cite{site:rfc9000,site:rfc9001,site:rfc9002}. 
\\\\
\emph{Quic-go} non è l'unica versione disponibile del protocollo \emph{QUIC}.
Come illustrato nella Figura \ref{table-implementazioni-quic}, esistono numerose altre implementazioni, sia \emph{open source} che proprietarie, ciascuna cerca di rispettare rigorosamente le specifiche definite negli \emph{RFC}. 
\\\\
La scelta di utilizzare questa specifica implementazione di \emph{QUIC} è motivata dal suo utilizzo all'interno di \textbf{Caddy}\footnote{\url{https://caddyserver.com/}}, un \emph{web server} moderno e performante. \emph{Caddy} integra \emph{Quic-go} per offrire il supporto nativo del protocollo \emph{QUIC} e \emph{HTTP/3}.
Inoltre, si è utilizzato \textbf{xCaddy}, un \emph{tool} che consente di creare build personalizzate di \emph{Caddy}, adattandolo specificatamente alle esigenze del progetto.

\begin{table}[!h]
    \centering
    \begin{tabular}{|c|c|c|}
        \hline
        \textbf{Nome} & \textbf{Linguaggio} & \textbf{Licenza} \\
        \hline
        Chromium & C++ & BSD License \\
        \hline
        MsQuic & C & MIT License \\
        \hline
        QUIC Library (mvfst) & C++ & MIT License \\
        \hline
        LiteSpeed QUIC Library (lsquic) & C & MIT License \\
        \hline
        ngtcp2 & C & MIT License \\
        \hline
        Quiche & Rust & BSD-2-Clause License \\
        \hline
        quicly & C & MIT License \\
        \hline
        \textbf{quic-go} & \textbf{Go} & \textbf{MIT License} \\
        \hline
        Quinn & Rust & Apache License 2.0 \\
        \hline
        Neqo & Rust & Apache License 2.0 \\
        \hline
        aioquic & Python & BSD-3-Clause License \\
        \hline
        picoquic & C & BSD-3-Clause License \\
        \hline
        pquic & C & MIT License \\
        \hline
        QUANT & C & BSD-2-Clause License \\
        \hline
        quic & Haskell & BSD-3-Clause License \\
        \hline
        netty-incubator-codec-quic & Java & Apache License 2.0 \\
        \hline
        nodejs-quic & NodeJs & MIT License \\
        \hline
        s2n-quic & Rust & Apache License 2.0 \\
        \hline
        swift-quic & Swift & Apache License 2.0 \\
        \hline
        TQUIC & Rust & Apache License 2.0 \\
        \hline
        nginx & C & BSD-2-Clause License \\
        \hline
        HAProxy & C & GNU General Public License version 2 \\
        \hline
        kwik & Java & GNU Lesser General Public License version 3 \\
        \hline
    \end{tabular}
    \caption{\emph{Tabella implementazioni QUIC}}
    \label{table-implementazioni-quic}
\end{table}

\subsection{Tecnologie Specifiche per MPTCP}
~\\
\indent Per quanto riguarda \emph{MPTCP} si è fatto riferimento alla documentazione ufficiale \cite{site:mptcp-code} e si è usato \emph{Go} per sviluppare un \emph{web server} che implementa un \emph{socket MPTCP}.
\\\\
Per garantire che le applicazioni utilizzassero \emph{MPTCP} si è usato \textbf{mptcpize}\footnote{\url{https://manpages.ubuntu.com/manpages/lunar/man8/mptcpize.8.html}}, un tool specifico che forza la creazione di \emph{socket MPTCP} al posto di quelli \emph{TCP}.
\hfill
\begin{table}[!h]
    \centering
    \begin{tabular}{|c|c|c|}
        \hline
        \textbf{Tipo} & \textbf{Nome} & \textbf{Versione} \\
        \hline
        Applicativo & \emph{Google Chrome} & 126.0.6478.126 \\
        \hline
        Applicativo & \emph{Firefox} & 127.0.2 \\
        \hline
        Applicativo & \emph{Github} &  \\
        \hline
        Applicativo & \emph{Git} & 2.43.0 \\
        \hline
        Applicativo & \emph{Oracle VM VirtualBox} & 7.0.20 \\
        \hline
        Applicativo & \emph{Wireshark} & 4.2.6 \\
        \hline
        Applicativo & \emph{Caddy} & 2.8.0 \\
        \hline
        Applicativo & \emph{xCaddy} & 0.4.2 \\
        \hline
        Applicativo & \emph{mptcpize} & 0.12 \\
        \hline
        Modulo & \emph{quic-go} &  0.43.1 \\
        \hline
        Sistema Operativo & \emph{Ubuntu} & 24.04 LTS \\
        \hline
        Linguaggio & \emph{Go} & 1.21 \\
        \hline
    \end{tabular}
    \caption{\emph{Tabella riassuntiva technologie usate}}
    \label{table-tecnologie}
\end{table}

\section{Esperimenti}
~\\
\indent La sezione corrente esamina in dettaglio gli esperimenti condotti nel corso dello studio, illustrando la logica sottostante e le procedure impiegate per la loro realizzazione. 
L'obiettivo è fornire una panoramica completa delle attività sperimentali, cosicchè ci sia una maggiore comprensione sia dei metodi utilizzati che degli scopi.
\\\\
Gli esperimenti sono stati creati con l'obiettivo di identificare e testare diversi metodi per aumentare il traffico dati e le ritrasmissioni.
Tutte le prove sono state condotte in un ambiente locale, simulando possibili scenari di rete tramite la creazione di specifici attori.
\\\\
I relativi risultati vengono analizzati nel Capitolo \ref{cap:risultati} mentre le conclusioni raggiunte sono discusse nel Capitolo \ref{cap:conclusioni}.

\subsection{Esperimenti QUIC}
~\\ 
\indent Gli esperimenti relativi al protocollo \emph{QUIC} si articolano in due principali categorie. 
La prima, che comprende il primo e secondo esperimento, avviene in uno scenario in cui uno degli attori della connessione, specificamente il server, assume un comportamento malevolo.
Il secondo si svolge in una situazione in cui l'attaccante non controlla direttamente nessuna delle due parti coinvolte nella comunicazione, ma si suppone sia in grado di monitorare e manipolare il traffico di rete.
\\\\
Per la realizzazione degli esperimenti si è utilizzato un \emph{fork} appositamente modificato del progetto \emph{Quic-go} \cite{site:my-fork}, 
con cui è stato possibile implementare il \emph{server} malevolo e condurre i vari test.
La simulazione degli attori dello scenario è stata eseguita utilizzando macchine virtuali con all'interno \emph{Ubuntu}.
Per quanto riguarda il \emph{client}, si sono utilizzati i \emph{browser} \emph{google-chrome} e \emph{firefox} per stabilire la connessione alla \emph{pagina web}. 
Sul lato \emph{server}, è stato impiegato \emph{Caddy} come \emph{web server}.

\paragraph{Esperimento 1: Server Web QUIC che ignora gli ACK}
\noindent L'idea alla base di questo esperimento è simulare un \emph{server web Quic} con un comportamento non convenzionale. 
Il \emph{server} è configurato per operare come se non ricevesse mai conferme \emph{(acknowledge)} per i pacchetti inviati, mantenendo al contempo un \emph{PTO} impostato a zero. 
Di conseguenza, il \emph{server} continua a ritrasmettere i pacchetti, presumendo erroneamente che non siano mai giunti a destinazione. Con il \emph{PTO} azzerato, queste ritrasmissioni avvengono in rapida successione, quasi simultaneamente.
\\\\
Per implementare queste modifiche, si é utilizzato il seguente approccio : 
\begin{enumerate}[label=\roman*]
    \item \textbf{Modifiche del codice sorgente di \emph{quic-go}}
    \begin{itemize}
        \item Si è modificata la gestione degli \emph{ACK} e la logica che gestisce il \emph{PTO}.
    \end{itemize}
    \item \textbf{Creazione di una build personalizzata di \emph{Caddy}}
    \begin{itemize}
        \item Si è utilizzato \emph{xCaddy} per la creazione di una versione modificata del \emph{server web Caddy} che usa il mio \emph{fork} di \emph{quic-go} con le modifiche apportate.
    \end{itemize}
\end{enumerate}
In particolare, le modifiche operano dopo la fase di \emph{handshake}, così da garantire l'inizializzazione della connessione. Successivamente, il \emph{server} agirà come modificato, continuando a ritrasmettere i vari dati.
\\\\
In questo scenario, ci si aspetta che il \emph{client} riconosca le ritrasmissioni e le segnali al \emph{server}. Eventuali segnali e richieste di chiusura della connessione sono per questo ignorate dal \emph{server}.
Data la natura \emph{end-to-end} delle connessioni \emph{QUIC}, eventuali nodi intermedi nella rete non sono in grado di valutare l'integrità o la correttezza di queste ritrasmissioni.
Ciò implica che solo il \emph{client} abbia la capacità di esaminare i pacchetti e scartarli e che quindi essi vengano contabilizzati nel suo traffico dati.
\\\\
Di seguito, viene riportata una descrizione di alcune delle modifiche effettuate al codice con le rispettive motivazioni.
\begin{lstlisting}[language=Go]
// internal/ackhandler/sent_packet_handler.go
const (
	// Maximum reordering in time space before time based loss detection considers a packet lost.
	// Specified as an RTT multiplier.
	timeThreshold = 9.0 / 8
	// Maximum reordering in packets before packet threshold loss detection considers a packet lost.
	packetThreshold = 3
	// Before validating the client's address, the server won't send more than 3x bytes than it received.
	amplificationFactor = 3
	// We use Retry packets to derive an RTT estimate. Make sure we don't set the RTT to a super low value yet.
	minRTTAfterRetry = 0 * time.Millisecond 
	// The PTO duration uses exponential backoff, but is truncated to a maximum value, as allowed by RFC 8961, section 4.4.
	maxPTODuration = 0 * time.Second 
)

func (h *sentPacketHandler) getScaledPTO(includeMaxAckDelay bool) time.Duration {
	// pto := h.rttStats.PTO(includeMaxAckDelay) << h.ptoCount
	pto := h.rttStats.PTO(includeMaxAckDelay)
	if pto > maxPTODuration || pto <= 0 {
		return maxPTODuration
	}
	return 0
	// return pto
}

// internal/utils/rtt_stats.go
func (r *RTTStats) PTO(includeMaxAckDelay bool) time.Duration {
	// if r.SmoothedRTT() == 0 {
	// 	return 2 * defaultInitialRTT
	// }
	// pto := r.SmoothedRTT() + max(4*r.MeanDeviation(), protocol.TimerGranularity)
	// if includeMaxAckDelay {
	// 	pto += r.MaxAckDelay()
	// }
	// return pto

	return 0
}
\end{lstlisting}
\noindent Il codice precedente elenca alcune modifiche effettuate per garantire che il \emph{PTO} rimanga costante a 0. 
In particolare, sono state alterate le costanti : 
\begin{itemize}
    \item \textbf{maxPTODuration} impostato a 0 e indica il massimo valore che il \emph{PTO} può assumere. 
    \item \textbf{minRTTAfterRetry} impostato a 0 e indica il valore minimo impostato dopo i pacchetti di \emph{retry}.
\end{itemize}
\noindent inoltre le funzioni che ricalcolano il \emph{PTO} sono state cambiate così che non lo modifichino.
\\
Di seguito, viene riportato il codice che impedisce al \emph{server} di riconoscere gli \emph{ACK} dei pacchetti inviati. 
Per ottenere questo risultato si è rimossa totalmente la logica che gestisce il ricevimento di un \emph{acknowledge} dalla funzione responsabile di questo e nel contempo ritorna sempre una tupla \emph{false|nil}, 
segnalando che nessun pacchetto è stato confermato.
\\
\begin{lstlisting}[language=Go]
// internal/ackhandler/sent_packet_handler.go
func (h *sentPacketHandler) ReceivedAck(ack *wire.AckFrame, encLevel protocol.EncryptionLevel, rcvTime time.Time) (bool /* contained 1-RTT packet */, error) { 
    pnSpace := h.getPacketNumberSpace(encLevel)
    largestAcked := ack.LargestAcked()
    if largestAcked > pnSpace.largestSent {
        fmt.Println("received ACK for an unsent packet")
        return false, &qerr.TransportError{
            ErrorCode:    qerr.ProtocolViolation,
            ErrorMessage: "received ACK for an unsent packet",
        }
    }
    // Servers complete address validation when a protected packet is received.
    if h.perspective == protocol.PerspectiveClient && !h.peerCompletedAddressValidation &&
        (encLevel == protocol.EncryptionHandshake || encLevel == protocol.Encryption1RTT) {
        h.peerCompletedAddressValidation = true
        h.logger.Debugf("Peer doesn't await address validation any longer.")
        // Make sure that the timer is reset, even if this ACK doesn't acknowledge any (ack-eliciting) packets.
        h.setLossDetectionTimer()
    }
    pnSpace.largestAcked = max(pnSpace.largestAcked, largestAcked)
    // Reset the pto_count unless the client is unsure if the server has validated the client's address.
	if h.peerCompletedAddressValidation {
		if h.tracer != nil && h.tracer.UpdatedPTOCount != nil && h.ptoCount != 0 {
			h.tracer.UpdatedPTOCount(0)
		}
		h.ptoCount = 0
	}
	h.numProbesToSend = 0
	if h.tracer != nil && h.tracer.UpdatedMetrics != nil {
		h.tracer.UpdatedMetrics(h.rttStats, h.congestion.GetCongestionWindow(), h.bytesInFlight, h.packetsInFlight())
	}
	h.setLossDetectionTimer()
	return false, nil
}
\end{lstlisting}
\noindent Ulteriori modifiche secondarie apportate, come la rimozione delle richieste di chiusura della comunicazione, sono visibili nel codice completo dell'esperimento \footnote{\url{https://github.com/GiovanniMenon/quic-go/tree/retransmission-1}} \cite{site:my-fork}.
\\\\
Partendo dall'idea di base di questo esperimento, sono state sviluppate diverse varianti ottenute modificando alcuni valori e parametri chiave. 
La lista completa è presente nell'Appendice. Le più significative di esse sono catalogate nella tabella \ref{table-retransmission}.
\begin{table}[!h]
    \centering
    \begin{tabular}{|c|c|c|}
        \hline
        \textbf{Nome} & \textbf{Connection Close} & \textbf{ACK Ignored} \\
        \hline
        retrasmission - 1  & Ignored & All \\
        \hline
        retrasmission - 2 & Not Ignored & All \\
        \hline
        retrasmission - 4 & Not Ignored & 1/2 \\
        \hline
        retrasmission - 5 & Ignored & 1/2 \\
        \hline
    \end{tabular}
    \caption{\emph{Tabella varianti esperimento 1}}
    \label{table-retransmission}
\end{table}

\paragraph{Esperimento 2: Server Web con Iniezione di Pacchetti in Background}
\noindent L'idea alla base di questo secondo esperimento è quella di simulare un \emph{server web QUIC} che, 
pur mantenendo un comportamento apparentemente normale durante la connessione, inietta in background pacchetti aggiuntivi non richiesti.
\\\\
In questo scenario, il \emph{server} inizializza e mantiene una connessione \emph{QUIC} standard con il \emph{client}, agendo normalmente. Tuttavia, in parallelo a questo, il \emph{server} è stato modificato appositamente per inviare pacchetti aggiuntivi non correlati alla comunicazione.
\\\\
Anche in questo esperimento si è utilizzato \emph{xCaddy} per la creazione di una versione modificata del \emph{server web Caddy} con le modifiche necessarie per iniettare silenziosamente i pacchetti aggiuntivi.
\\\\
L'operazione di iniezione avviene solo dopo la fase di \emph{handshake}, questa scelta è stata fatta per evitare potenziali interferenze, rallentamenti o errori nella prima fase della connessione. 
\\\\
Come nell'esperimento precedente, la natura \emph{end-to-end} della connessione \emph{QUIC} gioca un ruolo fondamentale. Si prevede che solo il \emph{client} sia in grado di verificare l'autenticità dei pacchetti ricevuti. 
Di conseguenza, questi finti pacchetti verranno contabilizzati nel traffico del client, nonostante non facciano parte della comunicazione.
\\\\
Questo particolare esperimento può essere considerato una variante del \emph{UDP flooding} \footnote{\gls{udp flooding}} \cite{site:udp-flood}, che sfrutta la natura \emph{end-to-end} del protocollo \emph{QUIC}.
\\\\
Di seguito, è riportata una descrizione delle modifiche apportate al codice sorgente per ottenere il risultato voluto.
\begin{lstlisting}[language=Go]
// sys_conn_oob.go
func (c *oobConn) WritePacket(b []byte, addr net.Addr, packetInfoOOB []byte, gsoSize uint16, ecn protocol.ECN) (int, error) {
    
    // ...
    // ...
    // ...

    if startSending {
            initBackgroundSender.Do(func() {
                const numWorkers = 6        // # Worker
                stop := make(chan struct{}) // Stop Channel

                go func() {
                    time.Sleep(backgroundInjectDuration)
                    close(stop)
                }()

                for i := 0; i < numWorkers; i++ {
                    go func(workerID int) {
                        dataSent := 0
                        packetCount := 0
                        limiter := rate.NewLimiter(rate.Limit(backgroundRateLimit), backgroundRateLimit)
                        for {
                            select {
                            case <-stop:
                                fmt.Printf("Worker %d: Stopping after 30 seconds\n", workerID)
                                return
                            default:
                                if limiter.Allow() {
                                    packetSize := rand.Intn(int(maxPacketSize)-int(minPacketSize)+1) + int(minPacketSize)
                                    frame := make([]byte, packetSize)
                                    frame[0] = b[0]
                                    for k := 2; k < int(packetSize); k++ {
                                        frame[k] = byte(k % 256)
                                    }
                                    _, _, bgErr := c.OOBCapablePacketConn.WriteMsgUDP(frame, oob, addr.(*net.UDPAddr))
                                    if bgErr != nil {
                                        fmt.Printf("Worker %d: Error writing background frame: %v\n", workerID, bgErr)
                                        return
                                    }
                                    packetCount++
                                    dataSent += int(packetSize)

                                    fmt.Printf("\r\tWorker %d: Frame %d sent, total data sent: %d bytes\n", workerID, packetCount, dataSent)
                                } else {
                                    time.Sleep(time.Millisecond * 100)
                                }
                            }
                        }
                    }(i)
                }
            })
        }
}
\end{lstlisting}
Per ottenere il risultato voluto, sono state apportate delle modifiche alla funzione \emph{WritePacket}. Si è introdotta una nuova logica che gestisce l'invio dei pacchetti aggiuntivi, 
tali modifiche includono: 
\begin{itemize}
    \item \textbf{Worker} : Si sono implementati \emph{N} worker che operano in parallelo per l'invio di dati. 
    \item \textbf{Limitatore}: È stato introdotto un meccanismo di limitazione per controllare il flusso di pacchetti iniettati (\emph{Packet/s}). 
    \item \textbf{PacketSize} : Ad ogni iterazione viene calcolata una \emph{packetSize} diversa. 
    \item \textbf{Packet} : La creazione del pacchetto si può dividere in un un processo specifico :
    \begin{itemize}
        \item Il payload è composto da una serie di \emph{frame} con dati fittizi.
        \item L'\emph{header} del pacchetto viene copiato, per praticità, da un pacchetto \emph{RTT-1} esistente. Questo assicura che l'\emph{header} sia coerente con la connessione.
    \end{itemize}
\end{itemize}
\noindent Infine si utilizza la funzione \emph{WriteMsgUDP} per costruire il pacchetto \emph{UDP} con il payload \emph{QUIC} ed inviarlo al destinatario \footnote{\url{https://github.com/GiovanniMenon/quic-go/tree/inject-1}}.
\\\\
Come nel primo esperimento, anche in questo caso sono state sviluppate diverse varianti che si distinguono principalmente per il numero di \emph{worker} impiegati e per le impostazioni del limitatore.
La lista completa è presente nell'Appendice. Le più significative sono elencate nella tabella \ref{table-inject}.
\begin{table}[!h]
    \centering
    \begin{tabular}{|c|c|c|}
        \hline
        \textbf{Nome} & \textbf{Worker} & \textbf{Packet/s} \\
        \hline
        inject - 1  & 6 & 1000 \\
        \hline
        inject - 2 & 6 & 2000 \\
        \hline
        inject - 3 & 8 & 1000 \\
        \hline
        inject - 4 & 8 & 2000 \\
        \hline
    \end{tabular}
    \caption{\emph{Tabella varianti esperimento 2}}
    \label{table-inject}
\end{table}
\paragraph{Esperimento 3: Manipolazione del traffico QUIC }
\noindent L'idea alla base di questo terzo esperimento si discosta da quelle precedenti, concentrandosi su uno scenario in cui non è presente un \emph{server} malevolo.
Viene invece simulata la presenza di un attaccante esterno in grado di manipolare il traffico in una connessione legittima tra \emph{client} e \emph{server}.
\\\\
In particolare, l'attaccante si concentra sull'oscuramento di specifici pacchetti all'interno della comunicazione \emph{QUIC}. 
L'obiettivo principale è quello di bloccare i pacchetti che con alta probabilità contengono un \emph{ACK}, facendo così credere al mittente che i suoi pacchetti non siano stati ricevuti.
Tuttavia, è importante sottolineare che, a causa della natura \emph{end-to-end} e della crittografia di \emph{QUIC}, non è possibile determinare quali pacchetti contengano effettivamente un \emph{ACK}.
\\\\
Nonostante questa caratteristica, dopo una serie di esperimenti, simulazioni e un'analisi di diverse connessioni \emph{QUIC}, si è osservato un pattern utile per fare inferenza sul contenuto dei pacchetti. Quelli contenenti un \emph{ACK frame} tendono ad avere una dimensione inferiore a \emph{85 bytes}.
È importante sottolineare che questa osservazione, pur non essendo sempre valida, offre comunque all'attaccante una base per selezionare euristicamente quali pacchetti oscurare.
\\\\
Per realizzare questo specifico scenario si è optato per una sua simulazione semplificata ma allo stesso tempo efficace. 
Invece di implementare un attaccante esterno separato, si è scelto di modificare il comportamento del \emph{server QUIC} per emulare il comportamento dell'attaccante.
Questo avviene programmando il \emph{server} per scartare automaticamente tutti i pacchetti, sia in ingresso che in uscita, con dimensione inferiore a \emph{85 bytes}.
Le modifiche sono state compiute aggiungendo il seguente codice sia alla funzione \emph{ReadPacket} che \emph{WritePacket} \footnote{\url{https://github.com/GiovanniMenon/quic-go/tree/spurious-1}}.
\begin{lstlisting}[language=Go]
// sys_conn_oob.go
if len(p.data)+42 < 85 {
	
	if probAckReceived%numberAckOscured == 0 {
		probAckReceived++
		return receivedPacket{}, nil
	}
	probAckReceived++
	return p, nil
}
\end{lstlisting}
Come negli esperimenti precedenti, sono state sviluppate diverse varianti che si differenziano per il numero di pacchetti che l'attaccante oscura. La lista completa è presente nell'Appendice. Le più rilevanti sono elencate nella tabella \ref{table-spurious}.
\begin{table}[!h]
    \centering
    \begin{tabular}{|c|c|}
        \hline
        \textbf{Nome} & \textbf{ACK Packet} \\
        \hline
        spurious - 1  & All \\
        \hline
        spurious - 2 & 1/2 \\
        \hline
        spurious - 3 & 2/3 \\
        \hline
        spurious - 4 & 4/5 \\
        \hline
    \end{tabular}
    \caption{\emph{Tabella varianti esperimento 3}}
    \label{table-spurious}
\end{table}
\pagebreak
\subsection{Esperimenti MPTCP}
~\\
\indent Gli esperimenti relativi al protocollo \emph{MPTCP} sono stati condotti in misura ridotta e si sono focalizzati su un unico scenario sperimentale. Per la realizzazione degli esperimenti si è creato un \emph{server web MPTCP} utilizzando i \emph{socket MPTCP} \cite{site:mptcp-code} e per la simulazione degli attori dello scenario sono state usate delle macchine virtuali con all'interno \emph{Ubuntu} e che supportano \emph{MPTCP}.
\paragraph{Esperimento 1: Server Web con Iniezione di Pacchetti in Background}
L'idea alla base di questo esperimento è analoga a quella del secondo esperimento di \emph{QUIC}, ma applicata al protocollo \emph{MPTCP}. 
L'obbiettivo è simulare un \emph{server web MPTCP} che, mantenendo all'apparenza un comportamento normale durante la connessione, inietta in background pacchetti aggiuntivi non richiesti.
\\\\
In questo scenario, il \emph{server} stabilisce e gestisce una connessione \emph{MPTCP} con il \emph{client}, ma parallelamente utilizza un \emph{raw socket} per iniettare pacchetti aggiuntivi nella comunicazione.
Una differenza significativa rispetto all'esperimento \emph{QUIC} è che \emph{MPTCP}, essendo un'estensione di \emph{TCP}, non include la crittografia \emph{end-to-end} integrata. 
Questo potrebbe influenzare il modo in cui i pacchetti iniettati vengono gestiti e rilevati lungo il percorso di rete.
\\\\
Come negli esperimenti precedenti, sono state sviluppate diverse varianti che si differenziano per il numero di \emph{worker} usati e per le impostazioni del limitatore. 
La lista completa è presente nella tabella \ref{table-inject-MPTCP}.
\begin{table}[!h]
    \centering
    \begin{tabular}{|c|c|c|}
        \hline
        \textbf{Nome} & \textbf{Worker} & \textbf{Packet/s} \\
        \hline
        inject - 1  & 6 & 1500 \\
        \hline
        inject - 2 & 6 & 3000 \\
        \hline
        inject - 3 & 8 & 1500 \\
        \hline
        inject - 4 & 8 & 3000 \\
        \hline
    \end{tabular}
    \caption{\emph{Tabella esperimento 1 MPTCP}}
    \label{table-inject-MPTCP}
\end{table}
