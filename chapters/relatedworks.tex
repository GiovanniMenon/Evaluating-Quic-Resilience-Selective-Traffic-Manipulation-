\chapter{Related Works}
\label{cap:RelatedWorks}

Sia \Citeauthor{article:handshake} che \Citeauthor{article:wild} , nei loro studi del 2017 e 2018, presentano una descrizione dettagliata
del protocollo \emph{QUIC} concentrandosi sulla sua scalabilità e sugli effetti che esso ha sulle reti moderne \cite{article:handshake,article:wild}.
Questi lavori affrontano in dettaglio le motivazioni principali che hanno portato allo sviluppo di \emph{QUIC} e analizzano come il protocollo sia stato progettato per 
supportare servizi su scala globale, come \emph{YouTube} e \emph{Google}. \\
\Citeauthor{article:securityAnaP}, nel loro studio, analizzano le performance di \emph{QUIC} evidenziando un compromesso tra la riduzione della latenza e la garanzia di sicurezza.
Gli autori, tramite un modello di sicurezza specifico, hanno dimostrato che il protocollo non garantisce la \emph{forward secrecy} tradizionale offerta da \emph{TLS}.
Lo studio mostra inoltre come una serie di attacchi semplici possa compromettere i vantaggi di \emph{QUIC}, 
mettendo in luce come i meccanismi utilizzati per aumentare la velocità del protocollo siano anche la causa di diverse debolezze.
\\
Gli studi, condotti nel 2022 e 2023, di \citeauthor{article:QuicAtt} e \Citeauthor{article:forge} offrono un'analisi più dettagliata e approfondita
della sicurezza del protocollo \emph{QUIC}, mettendo in luce come esso sia vulnerabile a numerosi attacchi e presenti diverse criticità in termini di sicurezza.
Nello studio di \Citeauthor{article:forge} vengono analizzate le vulnerabilità di \emph{QUIC} sugli attacchi di \emph{client-side request forgery}. Gli autori dimostrano, attraverso diverse modalità di \emph{request forgery},
che il protocollo è vulnerabile a fenomeni di amplificazione del traffico, nonostante i limiti anti-amplificazione previsti nella specifica \emph{RFC} del protocollo.
I risultati del lavoro di \citeauthor{article:QuicAtt} sono particolarmente rilevanti nella letteratura, 
in quanto hanno portato all'identificazione di diverse vulnerabilità \emph{zero-day} efficaci e pratiche. 
Gli autori hanno esaminato sei tra i più popolari \emph{server} compatibili con \emph{QUIC} utilizzando tecniche di \emph{fuzz testing} per l'identificazione delle vulnerabilità.
Questi test hanno evidenziato come le implementazioni a livello di produzione di \emph{QUIC} non siano ancora sufficientemente pronte per le reti moderne \cite{article:QuicAtt,article:forge}.
\\
Nel 2014, \citeauthor{article:cellular} ha pubblicato uno studio in cui vengono discusse alcune vulnerabilità delle reti mobili e della tariffazione del traffico dati 
per il protocollo \emph{TCP} \cite{article:cellular}. Questo lavoro ha evidenziato un importante problema di politiche relativo alla contabilizzazione dei dati da parte degli operatori.
Lo studio rivela che le ritrasmissioni \emph{TCP} possono essere facilmente sfruttate per manipolare la contabilizzazione del traffico cellulare. 
Questa scoperta non solo identifica una potenziale vulnerabilità nei sistemi di contabilizzazione del traffico dati,
ma dimostra anche come i diversi operatori adottino politiche diverse per il conteggio del traffico dati.
\\
A differenza degli studi citati, il nostro si concentra specificamente sul protocollo \emph{QUIC} nel contesto delle reti mobili e sulla contabilizzazione del traffico dati.
Il nostro studio prende spunto dal lavoro svolto da \citeauthor{article:cellular}, che ha analizzato 
come alcune vulnerabilità del protocollo \emph{TCP} possano essere sfruttate per manipolare la contabilizzazione del traffico dati nelle reti mobili cite{article:cellular}.
Applicando un approccio simile , si esaminano alcune potenziali strategie che sfruttano \emph{QUIC} per manipolare artificialmente il traffico dati.
Ci focalizziamo sulla possibilità che \emph{QUIC}, come \emph{TCP}, possa essere soggetto a problematiche di contabilizzazione.
