\chapter{Related Works}
\label{cap:RelatedWorks}

Sia \Citeauthor{article:handshake} che \Citeauthor{article:wild} , nei loro studi del 2017 e 2018, presentano una descrizione dettagliata
del protocollo \emph{QUIC} concentrandosi sulla sua scalabilità e sugli effetti che esso ha sulle reti moderne \cite{article:handshake,article:wild}.
Questi lavori affrontano in dettaglio le motivazioni principali che hanno portato allo sviluppo di \emph{QUIC} e analizzano come il protocollo sia stato progettato per 
supportare servizi su scala globale, come \emph{YouTube} e \emph{Google}. \\
\Citeauthor{article:securityAnaP}, nel loro studio, analizzano le performance di \emph{QUIC} in confronto ai protocolli \emph{UDP} e \emph{TCP} ed eseguono un'analisi sulla sua sicurezza 
per capirne i punti di forza e di depolezza \cite{article:securityAnaP}. Lo studio dimostra come \emph{QUIC} sia soggetto ad alcune problematiche legate alla \emph{data loss recovery}.
\\
Gli studi, condotti nel 2022 e 2023, di \citeauthor{article:QuicAtt} e \Citeauthor{article:forge} offrono un'analisi più dettagliata e approfondita
della sicurezza del protocollo \emph{QUIC}, mettendo in luce come esso sia vulnerabile a numerosi attacchi e presenti diverse criticità in termini di sicurezza.
I risultati del lavoro di \citeauthor{article:QuicAtt} sono particolarmente rilevanti nella letteratura, in quanto hanno portato all'identificazione di diverse vulnerabilità \emph{zero-day} efficaci e pratiche.
Queste vulnerabilità hanno evidenziato come le implementazioni a livello di produzione di \emph{QUIC} non siano ancora sufficientemente pronte per le reti moderne \cite{article:QuicAtt,article:forge}.
\\
Nel 2014, \citeauthor{article:cellular} ha pubblicato uno studio in cui vengono discusse alcune vulnerabilità delle reti mobili e della tariffazione del traffico dati 
per il protocollo \emph{TCP} \cite{article:cellular}. Questo lavoro evidenzia un importante problema di politiche relativo alla contabilizzazione dei dati da perte degli operatori.
Lo studio rivela che le ritrasmissioni \emph{TCP} possono essere facilmente sfruttate per manipolare la contabilizzazione del traffico cellulare. 
Questa scoperta non solo identifica una potenziale vulnerabilità nei sistemi di contabilizzazione del traffico dati,
ma dimostra anche come i diversi operatori adottino politiche diverse per il conteggio del traffico dati.
\\
Il nostro studio si propone di espandere il lavoro svolto da \citeauthor{article:cellular} applicando lo stesso approccio al protocollo \emph{QUIC}. 
In particolare, si esaminano le potenziali vulnerabilità di \emph{QUIC} e le possibili strategie per manipolare artificialmente il traffico dati di un utente nel contesto delle reti mobili. 