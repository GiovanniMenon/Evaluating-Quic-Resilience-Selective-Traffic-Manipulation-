\chapter{Related Works}
\label{cap:RelatedWorks}

Sia \Citeauthor{article:handshake} che \Citeauthor{article:wild} , nei loro studi del 2017 e 2018, presentano una descrizione dettagliata
del protocollo \emph{QUIC} concentrandosi sulla sua scalabilità e sugli effetti che esso ha sulle reti moderne \cite{article:handshake,article:wild}.
Questi lavori affrontano in dettaglio le motivazioni principali che hanno portato allo sviluppo di \emph{QUIC} e analizzano come il protocollo sia stato progettato per 
supportare servizi su scala globale, come \emph{YouTube} e \emph{Google}. \\
\Citeauthor{article:securityAnaP}, nel loro studio, analizzano le performance di \emph{QUIC} in confronto ai protocolli \emph{UDP} e \emph{TCP} ed eseguono un'analisi sulla sua sicurezza 
per capirne i punti di forza e di depolezza \cite{article:securityAnaP}. Lo studio dimostra come \emph{QUIC} sia soggetto ad alcune problematiche legate alla \emph{data loss recovery}.
\\
Gli studi, condotti nel 2022 e 2023, di \citeauthor{article:QuicAtt} e \Citeauthor{article:forge} offrono un'analisi più dettagliata e approfondita
della sicurezza del protocollo \emph{QUIC}, mettendo in luce come esso sia vulnerabile a numerosi attacchi e presenti diverse criticità in termini di sicurezza \cite{article:QuicAtt,article:forge}.
\\\\
Qui parlo di cosa ho fatto io e delle ricerche sulle problematiche delle reti mobili. 