\chapter{Introduzione}
\label{cap:introduzione}

(Bozza)\\\\
\emph{Abstract}- L'adozione di nuovi protocolli come \emph{Quick UDP Internet Connections (QUIC) } nelle reti moderne ha introdotto nuove sfide nella gestione e tariffazione del traffico.
Questo studio si concentra sull'analisi di questo protocollo con particolare attenzione alle problematiche relative alla tariffazione dei consumi e alle possibili tecniche per aumentare il traffico.
Si sono analizzate tre diverse strategie per la manipolazione del traffico: (1) un server malevolo che agisce come se non ricevesse alcun ACK e con un PTO costante a zero, (2) l'inserimento silenzioso di 
pacchetti aggiuntivi da parte del server durante una comunicazione, e (3) un attacco esterno che manipola il traffico mascherando selettivamente alcuni pacchetti.
Attraverso lo studio di queste strategie sono stati esaminati i risultati ottenuti. Lo studio ha evidenziato come queste strategie possano portare ad un aumento del traffico e 
dei consumi.


\section{Motivazione}
L'avvento di Internet e la continua evoluzione delle tecnologie di comunicazione ha trasformato radicalmente il modo di interagire e accedere alle informazioni.
Tuttavia, tale progresso porta con sè nuove sfide nella gestione e tariffazione del traffico.
I piani tariffari attuali, basati su soglie o sui consumi effettivi, si affidano al calcolo del consumo dati effettuato dagli operatori di rete.
Questo approccio, tuttavia, introduce una serie di nuove problematiche. Un esempio significativo è rappresentato 
dalla possibilità che un malintenzionato potrebbe sfruttare le caratteristiche intrinseche di un protocollo di rete o del sistema di comunicazione 
per manipolare artificialmente il consumo dati degli utenti senza che questi ne siano consapevoli.
Ciò comporta un aumento del costo o l'esaurimento prematuro delle risorse dati disponibili.
Risulta quindi necessario analizzare le possibili strategie utilizzate e valutare gli impatti che queste potrebbero avere sugli utenti.

\section{Organizzazione del testo}
\indent Di seguito, viene presentata la struttura del documento :
\begin{description}
    \item[{\hyperref[cap:RelatedWorks]{Il secondo capitolo}}] presenta quanto trovato di simile nella letteratura attuale;

    \item[{\hyperref[cap:descrizione]{Il terzo capitolo}}] approfondisce il background e illustra l'idea del progetto;
    
    \item[{\hyperref[cap:processi-metodologie]{Il quarto capitolo}}] descrive dettagliatamente l'ambiente di sviluppo e presenta i singoli esperimenti condotti;

    \item[{\hyperref[cap:risultati]{Il quinto capitolo}}] riassume e analizza i risultati ottenuti dagli esperimenti;
    
    \item[{\hyperref[cap:conclusioni]{Il sesto capitolo}}] presenta le conclusioni del lavoro e propone possibili sviluppi futuri.
\end{description}
