\chapter{Introduzione}
\label{cap:introduzione}

\textit{\indent In questo capitolo verrà presentata una introduzione del progetto, l'organizzazione del testo e l'insieme di convenzioni tipografiche adottate.
}
\section{Introduzione al Progetto}
(Ultima parte da fare)(abstract)\\
Qui Si parla delle reti mobile e di come negli ultimi anni siano cresciute e siano diventate molto diffuse. Si parla anche di come i protocolli di rete si sono evluti e di come ci si debba muovere per assicurare che non avvengano problemi durante questo tipo di connessioni. 


\section{Motivazione}
(Ultima parte da fare)(abstract)\\
(Perchè abbiamo deciso di andare a lavorare su questo tipo di protocollo e su questo scopo )
\section{Organizzazione del testo}

\indent Il documento è organizzato come segue :
\begin{description}
    \item[{\hyperref[cap:descrizione]{Il secondo capitolo}}] approfondisce il background e illustra l'idea del progetto;
    
    \item[{\hyperref[cap:processi-metodologie]{Il terzo capitolo}}] descrive dettagliatamente l'ambiente di sviluppo e presenta i singoli esperimenti condotti;

    \item[{\hyperref[cap:risultati]{Il quarto capitolo}}] riassume e analizza i risultati ottenuti dagli esperimenti;
    
    \item[{\hyperref[cap:conclusioni]{Il quinto capitolo}}] presenta le conclusioni del lavoro e propone possibili sviluppi futuri.
\end{description}

\subsection{Convenzioni Tipografiche}
Per la stesura del testo, sono state adottate le seguenti convenzioni tipografiche:
\begin{itemize}
    \item Gli acronimi, le abbreviazioni e i termini ambigui o di uso non comune sono definiti nella sezione Glossario.
    \begin{itemize} 
        \item Per la prima occorrenza dei termini riportati nel Glossario sarà presente una nota a piè di pagina con il relativo riferimento.
    \end{itemize}
	\item I termini in lingua straniera o facenti parti del gergo tecnico sono evidenziati con il carattere \emph{corsivo}.
	\item Eventuali fonti vengono riportate con un riferimento numerico alla relativa fonte nella sezione a fine documento.
	\begin{itemize} 
        \item Esempio di citazione \cite{site:agile-manifesto}.
        \item Esempio di citazioni multiple \cite{site:agile-manifesto,article:cellular}.
    \end{itemize}
\end{itemize}
