\chapter{Introduzione}
\label{cap:introduzione}

(Ultima parte da fare)(abstract)\\
Qui Si parla delle reti mobile e di come negli ultimi anni siano cresciute e siano diventate molto diffuse. Si parla anche di come i protocolli di rete si sono evluti e di come ci si debba muovere per assicurare che non avvengano problemi durante questo tipo di connessioni. 


\section{Motivazione}
(Ultima parte da fare)(abstract)\\
(Perchè abbiamo deciso di andare a lavorare su questo tipo di protocollo e su questo scopo )
\section{Organizzazione del testo}

\indent Di seguito, viene presentata la struttura del documento :
\begin{description}
    \item[{\hyperref[cap:RelatedWorks]{Il secondo capitolo}}] presenta quanto trovato di simile nella letteratura attuale;

    \item[{\hyperref[cap:descrizione]{Il terzo capitolo}}] approfondisce il background e illustra l'idea del progetto;
    
    \item[{\hyperref[cap:processi-metodologie]{Il quarto capitolo}}] descrive dettagliatamente l'ambiente di sviluppo e presenta i singoli esperimenti condotti;

    \item[{\hyperref[cap:risultati]{Il quinto capitolo}}] riassume e analizza i risultati ottenuti dagli esperimenti;
    
    \item[{\hyperref[cap:conclusioni]{Il sesto capitolo}}] presenta le conclusioni del lavoro e propone possibili sviluppi futuri.
\end{description}
