\chapter{Introduzione}
\label{cap:introduzione}

\section{Motivazione}
L'avvento di Internet e la continua evoluzione delle tecnologie di comunicazione ha trasformato radicalmente il modo in cui le persone interagiscono, lavorano e accedono alle informazioni.
Tuttavia, l'aumento esponenziale del traffico e la crescente complessità delle reti moderne comportano nuove sfide nella gestione e tariffazione del traffico.
Attualmente, le infrastrutture si basano su piani tariffari che dipendono da soglie prestabilite o sul consumo effettivo di dati, il cui calcolo è affidato agli operatori di rete.
Questo approccio, tuttavia, introduce una serie di nuove problematiche. Un esempio significativo è rappresentato 
dalla possibilità che un malintenzionato potrebbe sfruttare le caratteristiche intrinseche di un protocollo di rete o del sistema di comunicazione 
per manipolare artificialmente il consumo dati degli utenti senza che questi ne siano consapevoli.
Tale manipolazione comporta un aumento ingiustificato dei costi o l'esaurimento prematuro delle risorse dati disponibili.
In questo contesto, questo studio si propone di analizzare in dettaglio alcune delle possibili strategie che possono essere adottate dai malintenzionati, 
valutandone gli impatti potenziali e studiandone gli effetti sul consumo dati di una vittima. 
Questa analisi è importante per esplorare la portata delle minacce e i potenziali rischi connessi legati a questa tipologia di attacco.
\section{Organizzazione del testo}
\indent Di seguito, viene presentata la struttura del documento :
\begin{description}
    \item[{\hyperref[cap:RelatedWorks]{Il secondo capitolo}}] presenta quanto trovato di simile nella letteratura attuale;

    \item[{\hyperref[cap:descrizione]{Il terzo capitolo}}] approfondisce il background e illustra l'idea del progetto;
    
    \item[{\hyperref[cap:processi-metodologie]{Il quarto capitolo}}] descrive dettagliatamente l'ambiente di sviluppo e presenta i singoli esperimenti condotti;

    \item[{\hyperref[cap:risultati]{Il quinto capitolo}}] riassume e analizza i risultati ottenuti dagli esperimenti;
    
    \item[{\hyperref[cap:conclusioni]{Il sesto capitolo}}] presenta le conclusioni del lavoro e propone possibili sviluppi futuri.
\end{description}
