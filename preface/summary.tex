\cleardoublepage
\phantomsection
\pdfbookmark{Sommario}{Sommario}
\begingroup
\let\clearpage\relax
\let\cleardoublepage\relax
\let\cleardoublepage\relax

\chapter*{Sommario}

Il presente documento descrive il lavoro svolto durante il periodo di stage, della durata complessiva di trecentoventi ore, dal laureando Giovanni Menon presso l'Università degli studi di Padova.
Il tirocinio è stato condotto sotto la guida del Prof. Alessandro Galeazzi e con la collaborazione del Dott. Enrico Bassetti.
Il Prof. Davide Bresolin ha ricoperto il ruolo di tutor accademico e referente interno al Consiglio del Corso di Studio.
\\\\
Questa tesi riguarda l'analisi del protocollo \emph{\gls*{QUIC}} nel contesto delle reti moderne, 
con particolare attenzione alle problematiche legate alla tariffazione del traffico dati. Lo studio ha esplorato alcune possibili strategie che potrebbero
essere sfruttate per manipolare artificialmente il traffico mobile. 
I risultati ottenuti hanno evidenziato come queste strategie possano indurre un incremento del traffico dati per l'utente vittima, 
aumentando di conseguenza i suoi consumi dati e i suoi relativi costi. 
\\\\
Il tirocinio si è suddiviso in due parti.
La prima dedicata ad un'analisi approfondita del protocollo \emph{QUIC}, esaminando non solo lo stato dell'arte attuale e le tecnologie associate, ma anche la sua logica intrinseca e il suo funzionamento.
La seconda parte, invece, si è concentrata sulla progettazione e realizzazione degli esperimenti volti a testare le strategie identificate, terminando nella raccolta e nell'analisi dei risultati ottenuti.
%\vfill

%\selectlanguage{english}
%\pdfbookmark{Abstract}{Abstract}
%\chapter*{Abstract}

%\selectlanguage{italian}

\endgroup

\vfill
