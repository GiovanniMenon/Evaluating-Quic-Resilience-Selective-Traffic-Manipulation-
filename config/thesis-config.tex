% \omiss produces '[...]'
\newcommand{\omissis}{[\dots\negthinspace]}

% Itemize symbols
% see: https://tex.stackexchange.com/a/62497
% \renewcommand{\labelitemi}{$\bullet$}
% \renewcommand{\labelitemii}{$\cdot$}
% \renewcommand{\labelitemiii}{$\diamond$}
% \renewcommand{\labelitemiv}{$\ast$}


\let\Chaptermark\chaptermark
% \Chaptername gives current chapter name
\def\chaptermark#1{\def\Chaptername{#1}\Chaptermark{#1}}
\makeatletter
% \currentname gives the current section name
\newcommand*{\currentname}{\@currentlabelname}
\makeatother

% Uncomment the following line for a different header/footer style
% \pagestyle{fancy} \setlength{\headheight}{14.5pt}
\fancyhead[L]{\fontsize{12}{14.5} \selectfont \thechapter. \Chaptername}
\fancyhead[R]{\fontsize{12}{14.5} \selectfont \currentname}
% Page number always in footer
\cfoot{\thepage}


% Custom hyphenation rules
\hyphenation {
    e-sem-pio
    ex-am-ple
}

% Images path, not using \graphicspath because it doesn't properly work with
% latexmk custom dependencies
\NewCommandCopy{\latexincludegraphics}{\includegraphics}
\renewcommand{\includegraphics}[2][]{\latexincludegraphics[#1]{../images/#2}}

% Page format settings
% see: http://wwwcdf.pd.infn.it/AppuntiLinux/a2547.htm
\setlength{\parindent}{14pt}    % first row indentation
\setlength{\parskip}{0pt}       % paragraphs spacing


% Load variables
\newcommand{\myName}{Giovanni Menon}
\newcommand{\myID}{2034301}
\newcommand{\myTitle}{Valutazione della Resilienza di QUIC: Manipolazione Selettiva del Traffico}
\newcommand{\myDegree}{Tesi di laurea}
\newcommand{\myUni}{Università degli Studi di Padova}
% For BSc level just use "Corso di Laurea" and don't add "Triennale" to it
\newcommand{\myFaculty}{Corso di Laurea in Informatica}
\newcommand{\myDepartment}{Dipartimento di Matematica ``Tullio Levi-Civita''}
\newcommand{\profTitle}{Dott.}
\newcommand{\myProf}{Alessandro Galeazzi}
\newcommand{\myLocation}{Padova}
\newcommand{\myAA}{2023-2024}
\newcommand{\myTime}{Settembre 2024}

% PDF file metadata fields
% when updating them delete the build directory, otherwise they won't change
\begin{filecontents*}{\jobname.xmpdata}
  \Title{Document's title}
  \Author{Author's name}
  \Language{it-IT}
  \Subject{Short description}
  \Keywords{keyword1\sep keyword2\sep keyword3}
\end{filecontents*}


% Acronyms
\newacronym[description={\glslink{apig}{Application Program Interface}}]
    {api}{API}{Application Program Interface}

\newacronym[description={\glslink{umlg}{Unified Modeling Language}}]
    {uml}{UML}{Unified Modeling Language}

\newacronym[description={\glslink{ISO/OSIg}{Open System Interconnection}}]
    {ISO/OSI}{ISO/OSI}{Open System Interconnection}

\newacronym[description={\glslink{TCP}{Transmission Control Protocol}}]
    {TCP}{TCP}{Open System Interconnection}

\newacronym[description={\glslink{IoTg}{Internet of Things}}]
    {IoT}{IoT}{Internet of Things}

\newacronym[description={\glslink{UDP}{User Datagram Protocol}}]
    {UDP}{UDP}{User Datagram Protocol}

\newacronym[description={\glslink{ISN}{Initial Sequence Number}}]
    {ISN}{ISN}{Initial Sequence Number}

\newacronym[description={\glslink{RAN}{Radio Access Network}}]
    {RAN}{RAN}{Radio Access Network}
    
\newacronym[description={\glslink{RTT}{Round Trip Time}}]
    {RTT}{RTT}{Round Trip Time}

\newacronym[description={\glslink{RTO}{Retransmission TimeOut}}]
    {RTO}{RTO}{Retransmission TimeOut}

\newacronym[description={\glslink{VOIPg}{Voice Over IP}}]
    {VOIP}{VOIP}{Voice Over IP}

\newacronym[description={\glslink{SSL}{Secure Sockets Layer}}]
    {SSL}{SSL}{Secure Sockets Layer}

\newacronym[description={\glslink{TLS}{Transport Layer Security}}]
    {TLS}{TLS}{Transport Layer Security}

\newacronym[description={\glslink{QUIC}{Quic UDP Internet Connections}}]
    {QUIC}{QUIC}{Quic UDP Internet Connections}

\newacronym[description={\glslink{IETF}{Internet Engineering Task Force}}]
    {IETF}{IETF}{Internet Engineering Task Force}
\newacronym[description={\glslink{HTTP/3g}{HyperText Transfer Protocol/3}}]
    {HTTP/3}{HTTP/3}{HyperText Transfer Protocol/3}
\newacronym[description={\glslink{HTTPg}{HyperText Transfer Protocol}}]
    {HTTP}{HTTP}{HyperText Transfer Protocol}
% Glossary entries
\newglossaryentry{apig} {
    name=\glslink{api}{API},
    text=\emph{Application Program Interface},
    sort=api,
    description={in informatica con il termine \emph{Application Programming Interface API} (ing. interfaccia di programmazione di un'applicazione) si indica ogni insieme di procedure disponibili al programmatore, di solito raggruppate a formare un set di strumenti specifici per l'espletamento di un determinato compito all'interno di un certo programma. La finalità è ottenere un'astrazione, di solito tra l'hardware e il programmatore o tra software a basso e quello ad alto livello semplificando così il lavoro di programmazione}
}

\newglossaryentry{IoTg} {
    name=\glslink{IoT}{IoT},
    text=\emph{IoT},
    sort=IoT,
    description={Per IoT si intende l'internet delle cose, o meglio l'internet degli oggetti. Con questo termine si fa riferimento all'estensione di Internet al mondo degli oggetti e dei luoghi concreti. }
}

\newglossaryentry{protocolli di rete} {
    name=Protocolli di rete,
    text=\emph{Protocolli di rete},
    sort=protocolli di rete,
    description={Un protocollo di rete è l'insieme delle modalità o regole di relazione che due o più macchine collegate tra loro devono rispettare per stabilire una comunicazione corretta}
}

\newglossaryentry{protocolli di trasporto} {
    name=Protocolli di trasporto,
    text=\emph{Protocolli di trasporto},
    sort=protocolli di trasporto,
    description={Sono definiti protocolli di trasporto tutti i protocolli che appartengono al quarto livello del modello \emph{\gls{ISO/OSI}}. Ne fanno parte il \glslink{UDP}{UDP} e il \glslink{TCP}{TCP}. Hanno il compito di instaurare un collegamento logico tra le applicazioni residenti su macchine remote. }
}

\newglossaryentry{Handshake} {
    name=Handshake,
    text=\emph{Handshake},
    sort=Handshake,
    description={È il processo attraverso il quale due calcolatori negoziano e stabiliscono le regole comuni necessarie a stabilire una comunicazione}
}

\newglossaryentry{client} {
    name=Client,
    text=\emph{Client},
    sort=client,
    description={Il client è componente che accede a servizi o alle risorse forniti da un altro componente, detto \gls{server} }
}
\newglossaryentry{server} {
    name=Server,
    text=\emph{Server},
    sort=server,
    description={Un server è componente che mette a disposizione servizi o risorse ad altri componenti, detti \gls{client} un servizio}
}

\newglossaryentry{acknowledgements} {
    name=Acknowledgements,
    text=\emph{Acknowledgements},
    sort=acknowledgements,
    description={Da fare}
}
\newglossaryentry{checksum} {
    name=Checksum,
    text=\emph{Checksum},
    sort=checksum,
    description={Da fare}
}
\newglossaryentry{sliding window} {
    name=Sliding window,
    text=\emph{Sliding window},
    sort=sliding window,
    description={Da fare}
}
\newglossaryentry{multihoming} {
    name=Multihoming,
    text=\emph{Multihoming},
    sort=multihoming,
    description={Da fare}
}

\newglossaryentry{livello di presentazione} {
    name=Livello di presentazione,
    text=\emph{Livello di presentazione},
    sort=livello di presentazione,
    description={Da fare}
}

\newglossaryentry{ISO/OSIg} {
    name=\glslink{ISO/OSI}{ISO/OSI},
    text=\emph{ISO/OSI},
    sort=ISO/OSI,
    description={Da fare}
}

\newglossaryentry{VOIPg} {
    name=\glslink{VOIP}{VOIP},
    text=\emph{VOIP},
    sort=VOIP,
    description={Da fare}
}
\newglossaryentry{HTTP/3g} {
    name=\glslink{HTTP/3}{HTTP/3},
    text=\emph{HTTP/3},
    sort=HTTP/3,
    description={Da fare}
}
\newglossaryentry{HTTPg} {
    name=\glslink{HTTP}{HTTP},
    text=\emph{HTTP},
    sort=HTTP,
    description={Da fare}
}
\newglossaryentry{ossification} {
    name=Ossification,
    text=\emph{Ossification},
    sort=Ossification,
    description={Da fare}
}
\newglossaryentry{middle box} {
    name=Middle box,
    text=\emph{Middle box},
    sort=Middle box,
    description={Da fare}
}
\newglossaryentry{overhead} {
    name=Overhead,
    text=\emph{Overhead},
    sort=Overhead,
    description={Da fare}
}

\newglossaryentry{Connection ID} {
    name=Connection Identifier ,
    text=\emph{Connection Identifier },
    sort=Connection Identifier ,
    description={Da fare}
}

\newglossaryentry{livello applicativo} {
    name=Livello Applicativo ,
    text=\emph{Livello Applicativo },
    sort=Livello Applicativo ,
    description={Da fare}
}

\newglossaryentry{nonce crittografico} {
    name=Nonce Crittografico ,
    text=\emph{Nonce Crittografico },
    sort=Nonce Crittografico ,
    description={Da fare}
}

\newglossaryentry{endpoint} {
    name=Endpoint ,
    text=\emph{Endpoint},
    sort=Endpoint ,
    description={Da fare}
}
\makeglossaries

\bibliography{appendix/bibliography}

\defbibheading{bibliography} {
    \cleardoublepage
    \phantomsection
    \addcontentsline{toc}{chapter}{\bibname}
    \chapter*{\bibname\markboth{\bibname}{\bibname}}
}

% Spacing between entries
\setlength\bibitemsep{1.5\itemsep}

\DeclareBibliographyCategory{opere}
\DeclareBibliographyCategory{web}

\addtocategory{opere}{womak:lean-thinking}
\addtocategory{web}{site:agile-manifesto}

\defbibheading{opere}{\section*{Riferimenti bibliografici}}
\defbibheading{web}{\section*{Siti Web consultati}}


\captionsetup{
    tableposition=top,
    figureposition=bottom,
    font=small,
    format=hang,
    labelfont=bf
}

\hypersetup{
    %hyperfootnotes=false,
    %pdfpagelabels,
    colorlinks=true,
    linktocpage=true,
    pdfstartpage=1,
    pdfstartview=,
    breaklinks=true,
    pdfpagemode=UseNone,
    pageanchor=true,
    pdfpagemode=UseOutlines,
    plainpages=false,
    bookmarksnumbered,
    bookmarksopen=true,
    bookmarksopenlevel=1,
    hypertexnames=true,
    pdfhighlight=/O,
    %nesting=true,
    %frenchlinks,
    urlcolor=webbrown,
    linkcolor=RoyalBlue,
    citecolor=webgreen
    %pagecolor=RoyalBlue,
}

% Delete all links and show them in black
\if \isprintable 1
    \hypersetup{draft}
\fi

% Listings setup
\lstset{
    language=[LaTeX]Tex,%C++,
    keywordstyle=\color{RoyalBlue}, %\bfseries,
    basicstyle=\small\ttfamily,
    %identifierstyle=\color{NavyBlue},
    commentstyle=\color{Green}\ttfamily,
    stringstyle=\rmfamily,
    numbers=none, %left,%
    numberstyle=\scriptsize, %\tiny
    stepnumber=5,
    numbersep=8pt,
    showstringspaces=false,
    breaklines=true,
    frameround=ftff,
    frame=single
}

\definecolor{webgreen}{rgb}{0,.5,0}
\definecolor{webbrown}{rgb}{.6,0,0}

\newcommand{\sectionname}{sezione}
\addto\captionsitalian{\renewcommand{\figurename}{Figura}
                       \renewcommand{\tablename}{Tabella}}

\newcommand{\glsfirstoccur}{\ap{{[g]}}}

\newcommand{\intro}[1]{\emph{\textsf{#1}}}

% Risks environment
\newcounter{riskcounter}                % define a counter
\setcounter{riskcounter}{0}             % set the counter to some initial value

%%%% Parameters
% #1: Title
\newenvironment{risk}[1]{
    \refstepcounter{riskcounter}        % increment counter
    \par \noindent                      % start new paragraph
    \textbf{\arabic{riskcounter}. #1}   % display the title before the content of the environment is displayed
}{
    \par\medskip
}

\newcommand{\riskname}{Rischio}

\newcommand{\riskdescription}[1]{\textbf{\\Descrizione:} #1.}

\newcommand{\risksolution}[1]{\textbf{\\Soluzione:} #1.}

% Use case environment
\newcounter{usecasecounter}             % define a counter
\setcounter{usecasecounter}{0}          % set the counter to some initial value

%%%% Parameters
% #1: ID
% #2: Nome
\newenvironment{usecase}[2]{
    \renewcommand{\theusecasecounter}{\usecasename #1}  % this is where the display of
                                                        % the counter is overwritten/modified
    \refstepcounter{usecasecounter}             % increment counter
    \vspace{10pt}
    \par \noindent                              % start new paragraph
    {\large \textbf{\usecasename #1: #2}}       % display the title before the
                                                % content of the environment is displayed
    \medskip
}{
    \medskip
}

\newcommand{\usecasename}{UC}

\newcommand{\usecaseactors}[1]{\textbf{\\Attori Principali:} #1. \vspace{4pt}}
\newcommand{\usecasepre}[1]{\textbf{\\Precondizioni:} #1. \vspace{4pt}}
\newcommand{\usecasedesc}[1]{\textbf{\\Descrizione:} #1. \vspace{4pt}}
\newcommand{\usecasepost}[1]{\textbf{\\Postcondizioni:} #1. \vspace{4pt}}
\newcommand{\usecasealt}[1]{\textbf{\\Scenario Alternativo:} #1. \vspace{4pt}}

% Namespace description environment
\newenvironment{namespacedesc}{
    \vspace{10pt}
    \par \noindent  % start new paragraph
    \begin{description}
}{
    \end{description}
    \medskip
}

\newcommand{\classdesc}[2]{\item[\textbf{#1:}] #2}
