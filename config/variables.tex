\newcommand{\myName}{Giovanni Menon}
\newcommand{\myID}{2034301}
\newcommand{\myTitle}{Valutazione della Resilienza di QUIC: Manipolazione Selettiva del Traffico}
\newcommand{\myDegree}{Tesi di laurea}
\newcommand{\myUni}{Università degli Studi di Padova}
% For BSc level just use "Corso di Laurea" and don't add "Triennale" to it
\newcommand{\myFaculty}{Corso di Laurea in Informatica}
\newcommand{\myDepartment}{Dipartimento di Matematica ``Tullio Levi-Civita''}
\newcommand{\profTitle}{Prof.}
\newcommand{\myProf}{Alessandro Galeazzi}
\newcommand{\tutorTitle}{Dott.}
\newcommand{\myTutor}{Enrico Bassetti}
\newcommand{\myLocation}{Padova}
\newcommand{\myAA}{2023-2024}
\newcommand{\myTime}{Settembre 2024}

% PDF file metadata fields
% when updating them delete the build directory, otherwise they won't change
\begin{filecontents*}{\jobname.xmpdata}
  \Title{Valutazione della Resilienza di QUIC: Manipolazione Selettiva del Traffico}
  \Author{Giovanni Menon}
  \Language{it-IT}
  \Subject{Il presente documento descrive il lavoro svolto durante il periodo di stage, della durata
  complessiva di trecentoventi ore, dal laureando Giovanni Menon presso l’Università
  degli studi di Padova. Il tirocinio è stato condotto sotto la guida del Prof. Alessandro
  Galeazzi e con la collaborazione del Dott. Enrico Bassetti. Il Prof. Davide Bresolin ha
  ricoperto il ruolo di tutor accademico e referente interno al Consiglio del Corso di Studio}
  \Keywords{QUIC\sep MPTCP\sep Vulnerabilità}
\end{filecontents*}
