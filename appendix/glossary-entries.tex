% Acronyms
\newacronym[description={\glslink{apig}{Application Program Interface}}]
    {api}{API}{Application Program Interface}

\newacronym[description={\glslink{umlg}{Unified Modeling Language}}]
    {uml}{UML}{Unified Modeling Language}

\newacronym[description={\glslink{ISO/OSIg}{Open System Interconnection}}]
    {ISO/OSI}{ISO/OSI}{Open System Interconnection}

\newacronym[description={\glslink{TCP}{Transmission Control Protocol}}]
    {TCP}{TCP}{Open System Interconnection}

\newacronym[description={\glslink{IoTg}{Internet of Things}}]
    {IoT}{IoT}{Internet of Things}

\newacronym[description={\glslink{UDP}{User Datagram Protocol}}]
    {UDP}{UDP}{User Datagram Protocol}

\newacronym[description={\glslink{ISN}{Initial Sequence Number}}]
    {ISN}{ISN}{Initial Sequence Number}

\newacronym[description={\glslink{RAN}{Radio Access Network}}]
    {RAN}{RAN}{Radio Access Network}
    
\newacronym[description={\glslink{RTT}{Round Trip Time}}]
    {RTT}{RTT}{Round Trip Time}

\newacronym[description={\glslink{RTO}{Retransmission TimeOut}}]
    {RTO}{RTO}{Retransmission TimeOut}
\newacronym[description={\glslink{VOIPg}{Voice Over IP}}]
    {VOIP}{VOIP}{Voice Over IP}

% Glossary entries
\newglossaryentry{apig} {
    name=\glslink{api}{API},
    text=Application Program Interface,
    sort=api,
    description={in informatica con il termine \emph{Application Programming Interface API} (ing. interfaccia di programmazione di un'applicazione) si indica ogni insieme di procedure disponibili al programmatore, di solito raggruppate a formare un set di strumenti specifici per l'espletamento di un determinato compito all'interno di un certo programma. La finalità è ottenere un'astrazione, di solito tra l'hardware e il programmatore o tra software a basso e quello ad alto livello semplificando così il lavoro di programmazione}
}

\newglossaryentry{IoTg} {
    name=\glslink{IoT}{IoT},
    text=IoT,
    sort=IoT,
    description={Per IoT si intende l'internet delle cose, o meglio Internet degli oggetti. Con questo termine si fa riferimento all'estensione di Internet al mondo degli oggetti e dei luoghi concreti. }
}

\newglossaryentry{protocolli di rete} {
    name=Protocolli di rete,
    text=protocolli di rete,
    sort=protocolli di rete,
    description={Un protocollo di rete è l'insieme delle modalità o regole di relazione che due o più macchine collegate tra loro devono rispettare per stabilire una comunicazione corretta}
}

\newglossaryentry{protocolli di trasporto} {
    name=Protocolli di trasporto,
    text=protocolli di trasporto,
    sort=protocolli di trasporto,
    description={Sono definiti protocolli di trasporto tutti i protocolli che appartengono al quarto livello del modello \glslink{ISO/OSI}{ISO/OSI}. Ne fanno parte il \glslink{UDP}{UDP} e il \glslink{TCP}{TCP}. Hanno il compito di instaurare un collegamento logico tra le applicazioni residenti su macchine remote. }
}

\newglossaryentry{Handshake} {
    name=Handshake,
    text=Handshake,
    sort=Handshake,
    description={È il processo attraverso il quale due calcolatori negoziano e stabiliscono le regole comuni necessarie a stabilire una comunicazione}
}

\newglossaryentry{client} {
    name=Client,
    text=client,
    sort=client,
    description={Il client è componente che accede a servizi o alle risorse forniti da un altro componente, detto \gls{server} }
}
\newglossaryentry{server} {
    name=Server,
    text=server,
    sort=server,
    description={Un server è componente che mette a disposizione servizi o risorse ad altri componenti, detti \gls{client} un servizio}
}

\newglossaryentry{acknowledgements} {
    name=Acknowledgements,
    text=acknowledgements,
    sort=acknowledgements,
    description={Da fare}
}
\newglossaryentry{checksum} {
    name=Checksum,
    text=checksum,
    sort=checksum,
    description={Da fare}
}
\newglossaryentry{sliding window} {
    name=Sliding window,
    text=sliding window,
    sort=sliding window,
    description={Da fare}
}
\newglossaryentry{multihoming} {
    name=Multihoming,
    text=multihoming,
    sort=multihoming,
    description={Da fare}
}

\newglossaryentry{livello di presentazione} {
    name=Livello di presentazione,
    text=livello di presentazione,
    sort=livello di presentazione,
    description={Da fare}
}

\newglossaryentry{ISO/OSIg} {
    name=\glslink{ISO/OSI}{ISO/OSI},
    text=ISO/OSI,
    sort=ISO/OSI,
    description={Da fare}
}

\newglossaryentry{VOIPg} {
    name=\glslink{VOIP}{VOIP},
    text=VOIP,
    sort=VOIP,
    description={Da fare}
}