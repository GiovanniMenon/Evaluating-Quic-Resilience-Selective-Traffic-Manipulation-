% Acronyms
\newacronym[description={\glslink{apig}{Application Program Interface}}]
    {api}{API}{Application Program Interface}

\newacronym[description={\glslink{umlg}{Unified Modeling Language}}]
    {uml}{UML}{Unified Modeling Language}

\newacronym[description={\glslink{ISO/OSI}{Open System Interconnection}}]
    {ISO/OSI}{ISO/OSI}{Open System Interconnection}

\newacronym[description={\glslink{TCP}{Transmission Control Protocol}}]
    {TCP}{TCP}{Open System Interconnection}

\newacronym[description={\glslink{IoTg}{Internet of Things}}]
    {IoT}{IoT}{Internet of Things}

\newacronym[description={\glslink{UDP}{User Datagram Protocol}}]
    {UDP}{UDP}{User Datagram Protocol}

% Glossary entries
\newglossaryentry{apig} {
    name=\glslink{api}{API},
    text=Application Program Interface,
    sort=api,
    description={in informatica con il termine \emph{Application Programming Interface API} (ing. interfaccia di programmazione di un'applicazione) si indica ogni insieme di procedure disponibili al programmatore, di solito raggruppate a formare un set di strumenti specifici per l'espletamento di un determinato compito all'interno di un certo programma. La finalità è ottenere un'astrazione, di solito tra l'hardware e il programmatore o tra software a basso e quello ad alto livello semplificando così il lavoro di programmazione}
}

\newglossaryentry{IoTg} {
    name=\glslink{IoT}{IoT},
    text=IoT,
    sort=IoT,
    description={Per IoT si intende l'internet delle cose, o meglio Internet degli oggetti. Con questo termine si fa riferimento all'estensione di Internet al mondo degli oggetti e dei luoghi concreti. }
}

\newglossaryentry{protocolli di rete} {
    name=Protocolli di rete,
    text=protocolli di rete,
    sort=protocolli di rete,
    description={Un protocollo di rete è l'insieme delle modalità o regole di relazione che due o più macchine collegate tra loro devono rispettare per stabilire una comunicazione corretta}
}

\newglossaryentry{protocolli di trasporto} {
    name=Protocolli di trasporto,
    text=protocolli di trasporto,
    sort=protocolli di trasporto,
    description={Sono definiti protocolli di trasporto tutti i protocolli che appartengono al quarto livello del modello \glslink{ISO/OSI}{ISO/OSI}. Ne fanno parte il \glslink{UDP}{UDP} e il \glslink{TCP}{TCP}. Hanno il compito di instaurare un collegamento logico tra le applicazioni residenti su macchine remote. }
}

\newglossaryentry{Handshake} {
    name=Handshake,
    text=Handshake,
    sort=Handshake,
    description={È il processo attraverso il quale due calcolatori negoziano e stabiliscono le regole comuni necessarie a stabilire una comunicazione}
}
