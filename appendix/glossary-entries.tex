% Acronyms
\newacronym[description={\glslink{TCP}{Transmission Control Protocol}}]
    {TCP}{TCP}{Transmission Control Protocol}
\newacronym[description={\glslink{UDP}{User Datagram Protocol}}]
    {UDP}{UDP}{User Datagram Protocol}
\newacronym[description={\glslink{QUIC}{Quick UDP Internet Connections}}]
    {QUIC}{QUIC}{Quick UDP Internet Connections}
    \newacronym[description={\glslink{MPTCP}{Multipath TCP}}]
    {MPTCP}{MPTCP}{Multipath TCP}
\newacronym[description={\glslink{IoTg}{Internet of Things}}]
    {IoT}{IoT}{Internet of Things}

\newacronym[description={\glslink{ISN}{Initial Sequence Number}}]
    {ISN}{ISN}{Initial Sequence Number}

\newacronym[description={\glslink{RAN}{Radio Access Network}}]
    {RAN}{RAN}{Radio Access Network}
    
\newacronym[description={\glslink{RTT}{Round Trip Time}}]
    {RTT}{RTT}{Round Trip Time}

\newacronym[description={\glslink{RTO}{Retransmission Timeout}}]
    {RTO}{RTO}{Retransmission TimeOut}

\newacronym[description={\glslink{VOIPg}{Voice Over IP}}]
    {VOIP}{VOIP}{Voice Over IP}

\newacronym[description={\glslink{SSL}{Secure Sockets Layer}}]
    {SSL}{SSL}{Secure Sockets Layer}

\newacronym[description={\glslink{TLS}{Transport Layer Security}}]
    {TLS}{TLS}{Transport Layer Security}

\newacronym[description={\glslink{IETF}{Internet Engineering Task Force}}]
    {IETF}{IETF}{Internet Engineering Task Force}
\newacronym[description={\glslink{HTTP/3g}{HyperText Transfer Protocol/3}}]
    {HTTP/3}{HTTP/3}{HyperText Transfer Protocol/3}
\newacronym[description={\glslink{HTTPg}{HyperText Transfer Protocol}}]
    {HTTP}{HTTP}{HyperText Transfer Protocol}
\newacronym[description={\glslink{CHLO}{Inchoate Client Hello}}]
    {CHLO}{CHLO}{Inchoate Client Hello}
\newacronym[description={\glslink{REJ}{Reject}}]
    {REJ}{REJ}{Reject}
\newacronym[description={\glslink{SHLO}{Server Hello}}]
    {SHLO}{SHLO}{Server Hello}
\newacronym[description={\glslink{HoL}{Head of Line Blocking}}]
    {HoL}{HoL}{Head of Line Blocking}
\newacronym[description={\glslink{IANA}{Internet Assigned Number Authority}}]
    {IANA}{IANA}{Internet Assigned Number Authority}
\newacronym[description={\glslink{RFCp}{Request for Comments}}]
    {RFC}{RFC}{Request for Comments}
\newacronym[description={\glslink{PTO}{Probe Timeout}}]
    {PTO}{PTO}{Probe Timeout}
    
% Glossary entries

\newglossaryentry{IoTg} {
    name=\glslink{IoT}{IoT},
    text=\emph{IoT},
    sort=IoT,
    description={Per IoT si intende l'internet delle cose, o meglio l'internet degli oggetti. Con questo termine si fa riferimento all'estensione di Internet al mondo degli oggetti e dei luoghi concreti}
}

\newglossaryentry{Handshake} {
    name=Handshake,
    text=\emph{Handshake},
    sort=Handshake,
    description={È il processo attraverso il quale due calcolatori negoziano e stabiliscono le regole comuni necessarie a stabilire una comunicazione}
}

\newglossaryentry{client} {
    name=Client,
    text=\emph{Client},
    sort=client,
    description={Il client è componente che accede a servizi o alle risorse forniti da un altro componente, detto \gls{server}}
}
\newglossaryentry{server} {
    name=Server,
    text=\emph{Server},
    sort=server,
    description={Un server è componente che mette a disposizione servizi o risorse ad altri componenti, detti \gls{client}}
}

\newglossaryentry{acknowledgements} {
    name=Acknowledgements,
    text=\emph{Acknowledgements},
    sort=acknowledgements,
    description={egnale di riconoscimento utilizzato per indicare la corretta ricezione di un'informazione o di un pacchetto di dati in un sistema di comunicazione}
}
\newglossaryentry{checksum} {
    name=Checksum,
    text=\emph{Checksum},
    sort=checksum,
    description={Indica una sequenza di bit che viene associata al pacchetto trasmesso con lo scopo di verificare l'integrità di un dato o di un messaggio}
}
\newglossaryentry{sliding window} {
    name=Sliding window,
    text=\emph{Sliding window},
    sort=sliding window,
    description={È un metodo utilizzato per gestire il flusso di dati tra due entità in una comunicazione. Una finestra mobile rappresenta un intervallo di pacchetti o dati che possono essere inviati o ricevuti senza la necessità di una conferma per ogni singolo pacchetto.
    La finestra si sposta avanti man mano che i pacchetti vengono riconosciuti e confermati, ottimizzando l'utilizzo della banda e l'efficienza della trasmissione}
}
\newglossaryentry{multihoming} {
    name=Multihoming,
    text=\emph{Multihoming},
    sort=multihoming,
    description={Indica una tecnica con la quale un singolo nodo, come un server o un router, è connesso a più reti o provider di servizi Internet. In questo modo si ha maggiore flessibilità e disponibilità, poichè il nodo può continuare a comunicare anche se una delle connessioni fallisce}
}

\newglossaryentry{VOIPg} {
    name=\glslink{VOIP}{VOIP},
    text=\emph{VOIP},
    sort=VOIP,
    description={Indica una tecnologia che rende possibile effettuare una comunicazione analoga a quella che si potrebbe ottenere mediante la rete telefonica sfruttando la 
    comunicazione a Internet o una qualsiasi rete di telecomunicazioni che utilizzi il protocollo IP} 
}
\newglossaryentry{HTTP/3g} {
    name=\glslink{HTTP/3}{HTTP/3},
    text=\emph{HTTP/3},
    sort=HTTP/3,
    description={È la terza versione del protocollo \gls{HTTPg} basata sul protocollo \gls{QUIC}. Migliora le prestazioni riducendo la latenza e migliorando la gestione dei pacchetti persi grazie alla capacità di effettuare \gls{multiplexing}}
}
\newglossaryentry{HTTPg} {
    name=\glslink{HTTP}{HTTP},
    text=\emph{HTTP},
    sort=HTTP,
    description={È un protocollo a livello applicativo utilizzato per il trasferimento di dati sul web. HTTP è il protocollo di base per la trasmissione di pagine web e altre risorse tra un \gls{client} e un \gls{server}}
}
\newglossaryentry{ossification} {
    name=Ossification,
    text=\emph{Ossification},
    sort=Ossification,
    description={È un fenomeno per cui un protocollo di rete, una volta stabilito e ampiamente adottato, diventa rigido e difficile da modificare o estendere. Questo accade quando le implementazioni del protocollo e le infrastrutture si adattano e si ottimizzano per funzionare con le specifiche originali del protocollo, rendendo complesso o impossibile apportare cambiamenti senza compromettere la compatibilità}
}
\newglossaryentry{middle box} {
    name=Middle box,
    text=\emph{Middle box},
    sort=Middle box,
    description={Sono l'insieme di dispositivi di rete intermedi tra il \gls{client} e il \gls{server}}
}
\newglossaryentry{overhead} {
    name=Overhead,
    text=\emph{Overhead},
    sort=Overhead,
    description={Indica le risorse accessorie richieste in sovrappiù rispetto a quelle strettamente necessarie per portare a termine un determinato processo}
}

\newglossaryentry{nonce crittografico} {
    name=Nonce Crittografico ,
    text=\emph{Nonce Crittografico },
    sort=Nonce Crittografico ,
    description={Numero o valore arbitrario che ha un utilizzo unico. Viene utilizzato nei protocolli di autenticazione per asicurare che i dati scambiati nelle vecchie comunicazioni non possano essere riutilizzati}
}

\newglossaryentry{endpoint} {
    name=Endpoint ,
    text=\emph{Endpoint},
    sort=Endpoint ,
    description={Indica un dispositivo o un nodo di rete finale che si trova all'estremità di un canale di comunicazione. Può essere un \gls{client}, un \gls{server}, un dispositivo mobile e qualsiasi altro dispositivo che si collega a una rete per inviare o ricevere dati}
}
\newglossaryentry{Diffie-Hellman} {
    name=Diffie-Hellman,
    text=\emph{Diffie-Hellman},
    sort=Diffie-Hellman,
    description={È un protocollo crittografico che consente a due entità di stabilire una chiave condivisa e segreta utilizzando un canale di comunicazione insicuro. Non è richiesto uno scambio di informazioni in precedenza e la chiave ottenuta mediante questo protocollo può essere successivamente impiegata per cifrare le comunicazioni successive}
}
\newglossaryentry{stream} {
    name=Stream,
    text=\emph{Stream},
    sort=Stream,
    description={È un canele in cui scorrono i dati tra la sorgente e destinazione. Rappresenta una sequenza continua di dati che vengono trasmessi attraverso una rete. È un concetto fondamentale per comprendere come il multiplexing gestisce più flussi di dati simultaneamente}
}
\newglossaryentry{end-to-end} {
    name=End-to-End,
    text=\emph{End-to-End},
    sort=End-to-End,
    description={Tutte le operazioni di una comunicazione, quali operazioni crittografiche e gestione degli errori, devono venire eseguite nei nodi terminali della comunicazione e non nei nodi intermediari. 
    Di conseguenza questo tipo di caratteristica indica che solo i nodi terminali sono a conoscienza di eventuali dettagli della connessione}
}
\newglossaryentry{multipath} {
    name=Multipath,
    text=\emph{Multipath},
    sort=Multipath,
    description={Tecnologia o metodo utilizzato nelle reti di comunicazione per inviare dati attraverso più percorsi simultaneamente tra un punto di origine e una destinazione}
}
\newglossaryentry{fork} {
    name=Fork,
    text=\emph{Fork},
    sort=Fork,
    description={Nel contesto dello sviluppo del software un fork è una nuova versione di un progetto esistente, che in particolare utilizza lo codice sorgente originale e può evolversi in modo indipendente dal progetto iniziale}
}
\newglossaryentry{Go} {
    name=Go,
    text=\emph{Go},
    sort=Go,
    description={È un linguaggio di programmazione ad alto livello open source realizzato da Google, è noto anche come Golang. È stato progettato per ottimizzare i tempi di compilazione e per soddisfare le esigenze della programmazione concorrente}
}
\newglossaryentry{RFCp} {
    name=Request for Comments,
    text=\emph{Request for Comments},
    sort=Request for Comments,
    description={È un documento ufficiale pubblicato dalla \gls{IETF} che riporta informazioni o specifiche riguardanti nuove ricerche, innovazioni e metolodie dell'ambito informatico}
}
\newglossaryentry{udp flooding} {
    name=UDP flooding,
    text=\emph{UDP flooding},
    sort=UDP flooding,
    description={È un tipo di attacco informatic \emph{DoS (Denial of Service)} in cui l'attaccante invia una grande quantità di pacchetti \gls{UDP}{} a un \gls{server} o a un altro dispositivo di rete con l'intento di sovraccaricarlo e renderlo non disponibile agli utenti}
}